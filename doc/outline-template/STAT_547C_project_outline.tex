%%%%%%%%%%%%%%%%%%%%%%%%%%%%%%%%%%%%%%%%%%%%%%%%%%%%%%%%%%%%%%%%%%%%%%%%%%%%%%%%%%%%
% Template for STAT 547C Final Project Outline
% Author: Ben Bloem-Reddy <benbr@stat.ubc.ca>
% Date: Oct. 17, 2019
% Acknowledgments: ETH, Peter Orbanz, John Cunningham
%%%%%%%%%%%%%%%%%%%%%%%%%%%%%%%%%%%%%%%%%%%%%%%%%%%%%%%%%%%%%%%%%%%%%%%%%%%%%%%%%%%%

\documentclass[]{STAT_547C}
\usepackage{STAT_547C}
% NOTE: change the name and email address to your name in STAT_547C.sty

\usepackage{booktabs}
\usepackage{amsmath,amsthm,amssymb,amsfonts}
\usepackage{textcomp}

\usepackage[sorting=none,backend=biber,bibstyle=alphabetic,citestyle=alphabetic,giveninits=true,natbib=true]{biblatex}
\bibliography{../../ref/STAT_547C.bib} % add the title and location of your bibliography file

\begin{document}

% NOTE: You will replace the title below with your actual Title.
\makeGenericHeader{Stein\textquotesingle s method in machine learning}{Project Outline}
\vspace{-2cm}
%%%%%%%%%%%%%%%%%%%
\section{Title}

The working title of my project is \emph{Stein\textquotesingle s method in machine learning}.  

%%%%%%%%%%%%%%%%%%%
\section{Background}

Stein's method \citep{stein1972bound} is a technique that can quantify the error in the approximation of one distribution by another in a variety of metrics \citep{ross2011fundamentals}. The method is mainly used to derive bounds on approximations as well as convergence results in a variety of disciplines (see e.g.\cite{reinert2011short} for some areas of application). In recent years, Stein's method has been introduced in the machine learning literature. \cite{gorham2015measuring,liu2016kernelized,chwialkowski2016kernel,oates2017control} used the method to design goodness of fit and sample quality tests for sampling methods and \cite{liu2016stein,zhuo2017message} achieved better performance and speed ups in variational inference using a version of Stein's discrepancy \cite{liu2016kernelized}. I am interested in gaining an understanding of Stein's method and in conducting a more in-depth analysis of Stein's variational inference.

%%%%%%%%%%%%%%%%%%%
\section{Technical aspects}

The project will draw on technical aspects of the following areas: convergence, expectations, kernels and measures, variational inference.


%%%%%%%%%%%%%%%%%%%
\section{Literature}

The key references for this project are:

\begin{itemize}
  \item \cite{ross2011fundamentals}, provides a review fo Stein's method and examples for different metrics and distributions.
  \item \cite{liu2016stein} the paper introduces Stein's variational inference.
  \item \cite{zhuo2017message} improves the method of \citet{liu2016stein}.
  \item \cite{liu2016kernelized, chwialkowski2016kernel} are foundational works for \cite{liu2016stein} as they introduce Stein's discrepancy and the derivation of the optimal result minimizing the discrepancy. The results are then adopted in the variational inference context.
\end{itemize}


%%%%%%%%%%%%%%%%%%%
\section{Plan}

I will carry out this project with the following sequence of steps: 
\begin{enumerate}
  \item I will introduce Stein's method and derive some of the theoretical results needed as background for the understanding of Stein's variational inference.
  \item I will provide a more theoretical introduction to Reproducing Kernel Hilbert Space (RKHS).
  \item I will reformulate Stein's variational inference of \citet{liu2016stein} in a more rigorous probabilistic way. 
  \item If time allows, I will also discuss \citet{zhuo2017message} in similar terms. 
\end{enumerate}


%%%%%%%%%%%%%%%%%%%
\section{Why I'm interested in this topic}

I am interested in variational inference and Markov Chain Monte Carlo methods and it seems like Stein's method can provide improvements to existing methods in the literature. I hope a deeper understanding of the method can help me find new areas of research.

%%%%%%%%%%%%%%%%%%%
\printbibliography
\end{document}

